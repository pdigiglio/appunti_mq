\documentclass[
	a4paper,
	10pt,
%	titlepage,
	headinclude=off,
	footinclude=off,
	version = first,
	oneside, openany,
%	twoside, openright,
%	manychapters,
	draft,
]{scrbook}
\usepackage[utf8]{inputenc}
\usepackage[T1]{fontenc}
\usepackage[english, italian]{babel}

% imposto iwona come font senza grazie
\renewcommand{\sfdefault}{iwona}

\usepackage[]{xcolor}
\usepackage{MyMathDef}

%stile pagina
\usepackage{
	emptypage,
	indentfirst,
	microtype,%migliore spaziatura delle parole
	mparhack%per le note a margine
}

\usepackage{scrpage2}
\pagestyle{scrheadings}

% miniindici
\usepackage[tight,italian]{minitoc}

% tabelle e figure
\usepackage{
	rotating,
	booktabs,
	multirow,
	wrapfig,
	subfig
}

% mostra i label degli oggetti.
% Meglio commentare quando si compila con "final"
\usepackage{showkeys}%, showidx}

% per le unità di misura
\usepackage{siunitx}
\DeclareSIUnit{\atmosphere}{atm}
\sisetup{
%	list-final-separator = { \translate{and} },
%	list-pair-separator = { \translate{and} },
	range-phrase = { e },
	output-product = \cdot,
	bracket-unit-denominator = true,
	detect-all,
	detect-mode = false,
	exponent-product = \cdot,
	separate-uncertainty,
	sticky-per = true,
%	per-mode=symbol,
%	input-protect-tokens=\dots,
%	input-symbols=\dots
}

% per i grafici
\usepackage{
	tikz,
	pgfplots,
	tikz-3dplot,
	ifthen
}
\usetikzlibrary{
	patterns,
	3d,
	fadings,
	arrows
}
\usepgfplotslibrary{units}
\tikzset{
	dot/.style={
		fill=black,
		circle,
		maximum size=1pt
	},
	>=stealth
}
\pgfplotsset{
	compat=1.6,
	every axis/.append style={
		disabledatascaling,
		samples =100,
		enlargelimits = 0.05,
		axis x line = middle,
		axis y line = middle,
		legend style = {
			shade,
			bottom color = gray!20,
			top color = white,
			anchor =north % west
		},
	},
	% queste vengono attivate dopo che il grafico è stato reato,
	% quindi sovrascrivono le opzioni che sono state già date 
%	every axis plot post/.append style={
%		mark=none,
%		color=black,
%		solid
%	},
	% imposto delle opzioni per i grafici
	cycle list={
		black,
		mark=none,
	},
	every pin/.append style={
		font=\footnotesize
	},
}

%%%%%%%%%%%%%%%%%%%%%%%%%%%%
%
% Indice Analitico
%==========
%
\usepackage{makeidx, multicol}
\makeindex
%%
\let\orgtheindex\theindex
\let\orgendtheindex\endtheindex
\def\theindex{%
	\def\twocolumn{\begin{multicols}{2}}%
	\def\onecolumn{}%
	\clearpage
	\orgtheindex
}
\def\endtheindex{%
	\end{multicols}%
	\orgendtheindex
}
%%%%%%%%%%%%%%%%%%%%%%%%%%%


\usepackage[%
	printonlyused,%
	smaller%
]{acronym}

\definecolor{halfgray}{gray}{0.55} % chapter numbers will be semi transparent .5 .55 .6 .0
\definecolor{webgreen}{rgb}{0,.5,0}
\definecolor{webbrown}{rgb}{.6,0,0}
\definecolor{Maroon}{cmyk}{0, 0.87, 0.68, 0.32}
\definecolor{RoyalBlue}{cmyk}{1, 0.50, 0, 0}
%\definecolor{Black}{cmyk}{0, 0, 0, 0}

\usepackage{hyperref}
\hypersetup{%
	final,
    %draft,	% = no hyperlinking at all (useful in b/w printouts)
    colorlinks=true, linktocpage=true, pdfstartpage=3, pdfstartview=FitV,%
    % uncomment the following line if you want to have black links (e.g., for printing)
    %colorlinks=false, linktocpage=false, pdfborder={0 0 0}, pdfstartpage=3, pdfstartview=FitV,% 
    breaklinks=true, pdfpagemode=UseNone, pageanchor=true, pdfpagemode=UseOutlines,%
    plainpages=false, bookmarksnumbered, bookmarksopen=true, bookmarksopenlevel=1,%
    hypertexnames=true, pdfhighlight=/O,%nesting=true,%frenchlinks,%
    urlcolor=webbrown, linkcolor=RoyalBlue, citecolor=webgreen, pagecolor=RoyalBlue,%
    %urlcolor=Black, linkcolor=Black, citecolor=Black, %pagecolor=Black,%
}  

% abilita la ricerca di ° a documento finito
\usepackage{textcomp}
% per le frazioni in linea
\usepackage{xfrac}

\newcommand{\eu}{\ensuremath{\mathrm{e}}}% costante eulero
\newcommand{\iu}{\ensuremath{\mathrm{i}}}%unità immaginaria
\newcommand{\deriv}[3][]{\dfrac{\ud^{#1}#2}{\ud #3^{#1}}}% derivata totale - notazione leibniz
\newcommand{\pderiv}[3][]{\dfrac{\de^{#1}#2}{\de #3^{#1}}}% derivata parziale - notazione leibniz

% testo fittizio
\usepackage{lipsum}

% spaziatura normale dopo il punto
\frenchspacing
% frequenza di divisione in sillabe (default=100)
% numero più piccolo -> sillabazione più frequente
\pretolerance=5

\areaset[current]{\textwidth}{701pt}

% titolo
\title{
	\color{darkgray}{
		Appunti di\\
	}
	\scshape\color{Maroon}{
		meccanica quantistica
	}
}
\subtitle{
	\color{gray}{
		prof.~Alberto Zaffaroni
	}
}
\author{
%	\scshape{%
		Di Giglio, Paolo%
%			\and%
%		Legramandi, Andrea%
%	}%
}
\date{
	\small\sf\color{gray}{
		Ultimo aggiornamento:\\
		\today
	}
}
\usepackage{bm}
\usepackage{amsmath,amsthm}
\usepackage{mathtools}

% operatori
\newcommand{\op}[1]{\ensuremath{\widehat{#1}}}
\newcommand{\Adj}[1]{\ensuremath{{#1}^\dag}}
\newcommand{\adj}[1]{\Adj{\op{#1}}}
%complesso coniugato
\newcommand{\cc}[1]{\ensuremath{\overline{#1}}}

%miscellanea
\newcommand{\Hil}{\ensuremath{\mathscr{H}}}
\renewcommand{\H}{\op{H}}

\usepackage{natbib}


\frenchspacing
\pretolerance=5
\begin{document}

%\maketitle

	\chapter*{Elenco degli acronimi}

\begin{acronym}
	\acro{pdf}[PDF]{\foreignlanguage{english}{Probability Density Function}}
	\acro{sonc}[SONC]{Sistema Ortonormale Completo}
\end{acronym}

%\frontmatter % titolo, indice, introduzione, etc...

%	\tableofcontents

\mainmatter % capitoli
%	\thispagestyle{empty}
%	\part{\scshape\color{Maroon}{modulo i}}
%	\thispagestyle{empty}
		\chapter{Princìpi della Meccanica Quantistica}
\label{chap:principi}

\Princ{Per ogni istante di tempo $t_0\in\R$ un sistema fisico è descritto da un vettore di stato $\ket{\psi(t_0)}$ in uno spazio di Hilbert \Hil{} (generalmente di dimensione non finita).}

\Princ{Ad ogni grandezza fisica $\mathcal{A}$ è associato un operatore autoaggiunto $\op{A}\colon\Hil\to\Hil\mid\op{A}=\adj{A}$.}

\noindent
Gli operatori \op{A} e \op{B} sono uguali quando hanno lo stesso dominio e $\op{A}\ket{\psi}=\op{B}\ket{\psi}$ per ogni $\ket{\psi}\in\mathscr{D}(\op{A})\seq\Hil$, dove $\mathscr{D}(\op{A})$ è l'insieme di definizione di \op{A}.

\Princ{I possibili risultati di una misura della grandezza $\mathcal{A}$ sul sistema coincidono con lo spettro dell'operatore \op{A}.}

\Princ{\label{princ:4}Sia $a_n$ l'autovalore (assunto non degenere) di \op{A} associato all'autofunzione $\ket{\varphi_n}$ di modo che $\op{A}\ket{\varphi_n}=a_n\ket{\varphi_n}$.
La probabilità che una misura di $\mathcal{A}$ su un sistema descritto dalla funzione d'onda $\ket{\psi(t)}$ restituisca $a_n$ è
\begin{equation}\label{princ:eq.1}
\p{\mathcal{A}=a_n;t}=\abs{\braket{\varphi_n|\psi(t)}}^2.
\end{equation}
}

\noindent
Nel caso in cui $a_n$ sia un autovalore $g_n$ volte degenere con autofunzioni associate $\ket{\varphi_n^i}$ la~\eqref{princ:eq.1} si può generalizzare come segue:
\begin{equation}
\p{\mathcal{A}=a_n;t}=\sum_{i=1}^{g_n}\abs{\braket{\varphi_n^i|\psi(t)}}^2.
\end{equation}


Nel caso di autovalori continui non si può parlare propriamente di probabilità di ottenere certo autovalore $\alpha$ ma è più corretto riferirsi alla probabilità che la misurazione restituisca un valore compreso tra $\alpha$ e $\alpha+\ud\alpha$.
È chiaro che $\p{\alpha;t}$ non è più una vera probabilità ma una densità di probabilità o \emph{\ac{pdf}}\index{\acl{pdf}}, di modo che
\begin{equation}
\p{\alpha<\mathcal{A}<\alpha+\ud\alpha;t}=\ud\p{\alpha;t}\coloneqq \abs{\braket{\alpha|\psi(t)}}^2\!\ud\alpha.
\end{equation}


\Princ{\label{princ:5}Se una misura della grandezza $\mathcal{A}$ su un sistema descritto da $\ket{\psi(t)}$ ha dato $a_n$, ogni misura immediatamente successiva della stessa grandezza restituirà il medesimo risultato.}

\noindent
Per ""immediatamente successivo'' s'intende che la nuova misura dev'essere effettuata prima che il sistema abbia avuto il tempo di evolvere.


Ora, il principio~\ref{princ:5} implica che dopo una misurazione il vettore di stato ha subìto un mutamento.
Sia, ad esempio, $\uD t>0$ un piccolo intervallo di tempo e $\ket{\psi_1}\coloneqq\ket{\psi(t_0-\uD t)}$ il vettore d'onda precedente alla misura --- effettuata al tempo $t_0$.
Se all'autovalore $a_n$ è associato il sottospazio $\Hil_n\coloneqq \Span\Set{ \Ket{\varphi_n^i} }$ con $i\in\Set{1,\dots,g_n}$, essendo $g_n$ il grado di degenerazione di $a_n$, avremo
\begin{equation}
\ket{\psi(t_0+\uD t)}=
\frac{\displaystyle \sum_{i=1}^{g_n}\ket{\varphi_n^i}\!\braket{\varphi_n^i|\psi_1}}%(t_0-\uD t)}}%
{\sqrt{\displaystyle \sum_{j=1}^{g_n}\abs{\braket{\varphi_n^j|\psi_1}}^2}}.%(t_0-\uD t)}}^2}}.
\end{equation}
La misurazione di una grandezza ha dunque introdotto un mutamento discontinuo nell'evoluzione temporale della funzione d'onda che \emph{precipita} nell'autostato (o in una combinazione di autostati nel caso degenere) associato ad $a_n$.

\Princ{\label{princ:6}L'evoluzione temporale del vettore di stato $\ket{\psi(t)}\in\Hil$ è retta dall'equazione di Schr\"odinger
\begin{equation}\label{princ:eq.2}
\iu\hbar\,\deriv{}{t}\ket{\psi(t)}=\H(t)\ket{\psi(t)},\tag{SE}
\end{equation}
dove $\H(t)$ è l'\emph{hamiltoniana}\index{operatore!hamiltoniano} del sistema.}

\noindent
L'operatore $\H(t)$ si ottiene dall'hamiltoniana classica $H(x_i,p_i,t)$ rimpiazzando le variabili coniugate $\vec{x}$ e \vec{p} con le loro controparti quantistiche.
In generale bisognerà garantire l'hermitianicità di \H{} dal momento che, pur essendo \op{x_i} e \op{p_i} autoaggiunti, non è detto che $f(\op{x_i},\op{p_i})$ lo sia a sua volta.


Basti considerare la semplice combinazione $f(\op{x_i},\op{p_i})\coloneqq\vec{\op{x}}\cdot\vec{\op{p}}=\op{x_i}\,\op{p_i}$ (gli indici ripetuti s'intendono sommati) per la quale
\begin{equation}
\Adj{f}=\Adj{(\op{x_i}\,\op{p_i})}=\adj{p_i}\adj{x_i}=\op{p_i}\,\op{x_i}\neq\op{x_i}\,\op{p_i}=f.
\end{equation}
È chiaro, tuttavia, che l'operatore $\tilde{f}(\op{x_i},\op{p_i})\coloneqq(\op{x_i}\,\op{p_i}+\op{p_i}\,\op{x_i})/2$ è autoaggiunto.
In generale sarà possibile rendere hermitiane le nostre funzioni $f$ ricorrendo a qualche accorgimento.


\section{Flusso di probabilità}

Consideriamo l'equazione di Schr\"odinger~\eqref{princ:eq.2} sviluppata sul \ac{sonc} formato dalle autofunzioni (generalizzate) dell'operatore \op{x}
\begin{equation}
\iu\hbar\,\pderiv{}{t}\psi(\vec{x},t)=-\frac{\hbar^2}{2m}\lap\psi(\vec{x},t)+V(\vec{x},t)\,\psi(\vec{x},t)
\end{equation}
e la sua complessa coniugata
\begin{equation}
-\iu\hbar\,\pderiv{}{t}\cc\psi(\vec{x},t)=-\frac{\hbar^2}{2m}\lap\cc\psi(\vec{x},t)+V(\vec{x},t)\,\cc\psi(\vec{x},t).
\end{equation}
Sia ora $\rho(\vec{x},t)\coloneqq \abs{\psi(\vec{x},t)}^2$ la densità volumetrica di probabilità.
\begin{align*}
\pderiv{\rho}{t}&=\cc\psi\,\pderiv{\psi}{t}+\psi\,\pderiv{\cc\psi}{t}=-\hbar\,\frac{\cc\psi\lap\psi-\psi\lap\cc\psi}{2\iu m}\\
&=-\di\tp*{\hbar\,\frac{\cc\psi\grad\psi-\psi\grad\cc\psi}{2\iu m}}\eqqcolon-\di\vec{J}.
\end{align*}
Se definisco il \emph{flusso} o \emph{corrente di probabilità}
\begin{equation}
\vec{J}(\vec{x},t) \coloneqq \hbar\,\frac{\cc\psi\grad\psi-\psi\grad\cc\psi}{2\iu m},
\end{equation}
ottengo l'equazione di continuità
\begin{empheq}{equation}
\pderiv{\rho}{t}+\di\vec{J}=0.
\end{empheq}


Sia dato un volume $\Omega\seq\R^3$ di cui indicheremo il bordo con $\de\Omega$.
\begin{equation}
\deriv{}{t}\int_\Omega\rho\ud^3x=
\int_\Omega\pderiv{\rho}{t}\ud^3x
=-\int_\Omega\di\vec{J}\ud^3x=
-\int_{\de\Omega}\vec{J}\cdot\ver{n}\ud^2x.
\end{equation}
Se $\Omega\to\R^3$ la probabilità nello spazio si conserva.
Ma il vettore inizialmente assunto normalizzato dunque dev'essere
\begin{equation}
\int_{\R^3}\abs{\psi(\vec{x},t)}\ud^3x\equiv 1.
\end{equation}


	\subsection{Teorema di Ehrenfest}

Sia dato un operatore $\op{A}(t)$ in generale dipendente (esplicitamente) dal tempo.
Si definisce 	\emph{valor medio}\index{valor medio} di $\op{A}(t)$ la quantità
\begin{equation}\label{princ:eq.3}
\braket{\op{A}(t)}\coloneqq\bra{\psi(t)}\op{A}(t)\ket{\psi(t)}\!.
\end{equation}
La definizione~\eqref{princ:eq.3} coincide con il tradizionale valore di aspettazione
\begin{equation}
\braket{\op{A}(t)}=\sum_k a_k\,\p{\mathcal{A}=a_k}+
\int\alpha\,\p{\mathcal{A}=\alpha}\ud\alpha.
\end{equation}
Per convincersene, si consideri un \ac{sonc} $\Set{\Ket{\varphi_k},\Ket{\kappa}}$ formato da autofunzioni {\color{red}discrete e continue} dell'operatore hamiltoniano.
Ogni funzione è sviluppabile come
\begin{equation}
\begin{aligned}
\ket{\psi(t)}&=\sum_k\ket{\varphi_k}\!\braket{\varphi_k|\psi(t)}+
\int \ket{\kappa}\!\braket{\kappa|\psi(t)}\ud \kappa\\
&=\sum_k\eu^{-\iu E_kt/\hbar}\ket{\varphi_k}\!\braket{\varphi_k|\psi(0)}+
\int \eu^{-\iu E_\kappa t/\hbar}\ket{\kappa}\!\braket{\kappa|\psi(0)}\ud \kappa
\end{aligned}
\end{equation}

%\backmatter %indice analitico, bibliografia...
	\printindex

%	\nocite{*}
%	\bibliographystyle{plain}
%	\bibliography{./bibliografia}
\end{document}